% Copyright (c) 2017, Gabriel Hjort Blindell <ghb@kth.se>
%
% This work is licensed under a Creative Commons 4.0 International License (see
% LICENSE file or visit <http://creativecommons.org/licenses/by/4.0/> for a copy
% of the license).



%======
% TEXT
%======

\newcommand{\labelFigure}[1]{\label{fig:#1}}
\newcommand{\refFigure}[1]{\mbox{Fig.\thinspace\ref{fig:#1}}}
\newcommand{\labelSection}[1]{\label{sec:#1}}
\newcommand{\refSection}[1]{\mbox{Sect.\thinspace\ref{sec:#1}}}
\newcommand{\labelEquation}[1]{\label{eq:#1}}
\NewDocumentCommand{\refEquation}{sm}{%
  \IfBooleanTF{#1}{%
    \ref{eq:#2}%
  }{%
    \mbox{Eq.\thinspace\ref{eq:#2}}%
  }%
}
\NewDocumentCommand{\citeEquation}{sm}{%
  \IfBooleanTF{#1}{%
    #2%
  }{%
    \mbox{Eq.\thinspace#2}%
  }%
}
\newcommand{\wrt}{w.r.t.}
\newcommand{\todo}[1]{\textcolor{black!25!red}{[#1]}}

% Enable bold version of the monotype font
% See http://www.macfreek.nl/memory/LaTeX_Bold_Typewriter_Font
\DeclareFontShape{OT1}{cmtt}{bx}{n}%
  {<5><6><7><8><9><10><10.95><12><14.4><17.28><20.74><24.88>cmttb10}{}

% For highlighting text.
% The strut macro was taken from https://tex.stackexchange.com/a/74469/2634
\newcommand{\hlDiffColor}{black!25}
\newcommand{\hlStrut}{%
  \vrule width 0pt height .9\ht\strutbox depth .9\dp\strutbox\relax%
}
\NewDocumentCommand{\hlDiff}{mo}{%
  \begingroup%
  \setlength{\fboxsep}{0pt}%
  \ifmmode%
    \IfValueT{#2}{\hspace{#2}}%
    \colorbox{\hlDiffColor}{\hlStrut$#1$}%
    \IfValueT{#2}{\hspace{#2}}%
  \else%
    \IfValueT{#2}{\hspace{#2}}%
    \colorbox{\hlDiffColor}{\hlStrut#1}%
    \IfValueT{#2}{\hspace{#2}}%
  \fi%
  \endgroup%
}



%======
% MATH
%======

% General commands
\newcommand{\overbar}[1]{
  \mkern 1.5mu\overline{\mkern-1.5mu#1\mkern-1.5mu}\mkern 1.5mu
}
\NewDocumentCommand{\mPowerset}{m}{
  2^{#1}
}\newcommand{\mNatNumSet}{\mathbb{N}}
\newcommand{\mPhi}{\varphi}
\newcommand{\mSet}[1]{\left\{ #1 \right\}}
\newcommand{\mCard}[1]{\left| #1 \right|}
\newcommand{\mEmptySet}{\varnothing}
\newcommand{\mFunFont}[1]{\textit{#1}}
\newcommand{\mTuple}[2]{\left\langle #1, #2 \right\rangle}
\newcommand{\mUnDirEdge}[2]{\mTuple{#1}{#2}}
\newcommand{\mAnd}{\wedge}
\newcommand{\mOr}{\vee}
\newcommand{\mImp}{\Rightarrow}
\newcommand{\mEq}{\Leftrightarrow}
\newcommand{\mQuantSep}{\quad}

% Constraint model-related commands
\newcommand{\mBrPattern}{g_{\mathrm{br}}}
\newcommand{\mKill}{\times}
\newcommand{\mNull}{\bot}
\newcommand{\mUPGraph}{G}
\newcommand{\mPatternSet}{S}
\newcommand{\mExtPatternSet}{\mPatternSet_\mathrm{ext}}
\newcommand{\mOpSet}{O}
\newcommand{\mOperandSet}{P}
\newcommand{\mForbiddenCombSet}{F}
\newcommand{\mCostMatrix}{\mathbf{C}}
\NewDocumentCommand{\mDataSet}{o}{
  \IfValueTF{#1}{D_{#1}}{D}
}
\newcommand{\mBlockSet}{B}
\newcommand{\mDefEdgeSet}{E}
\NewDocumentCommand{\mMatchSet}{o}{
  \IfValueTF{#1}{M_{#1}}{M}
}
\NewDocumentCommand{\mMatchCompSet}{m}{
  M_{\overbar{#1}}
}
\newcommand{\mLocationSet}{L}
\newcommand{\mFallThroughSet}{J}
\newcommand{\mNullLocation}{l_\mathrm{null}}
\NewDocumentCommand{\mVar}{mo}{
  \mathbf{#1}{\IfValueTF{#2}{[#2]}{}}
}
\newcommand{\mLlvmCost}{C_{\mathrm{LLVM}}}
\newcommand{\mRelaxedCost}{C_{\mathrm{relaxed}}}
\DeclareMathOperator{\mCircuit}{\mFunFont{circuit}}
\DeclareMathOperator{\mConsumes}{\mFunFont{consumes}}
\DeclareMathOperator{\mCost}{\mFunFont{cost}}
\DeclareMathOperator{\mCovers}{\mFunFont{covers}}
\DeclareMathOperator{\mDefines}{\mFunFont{defines}}
\DeclareMathOperator{\mDom}{\mFunFont{dom}}
\DeclareMathOperator{\mEmptyBlock}{\mFunFont{empty}}
\DeclareMathOperator{\mEntry}{\mFunFont{entry}}
\DeclareMathOperator{\mFreq}{\mFunFont{freq}}
\DeclareMathOperator{\mIntValues}{\mFunFont{intvalues}}
\DeclareMathOperator{\mOpCost}{\mFunFont{cost}}
\DeclareMathOperator{\mSpans}{\mFunFont{spans}}
\DeclareMathOperator{\mStores}{\mFunFont{stores}}
\DeclareMathOperator{\mTable}{\mFunFont{table}}
\DeclareMathOperator{\mUses}{\mFunFont{uses}}



%============
% REFERENCES
%============

\addbibresource{references.bib}

% Command alias.
\NewDocumentCommand{\printreferences}{}{%
  \printbibliography%
}

% Bibliography options:
%    - Write first and middle names as initials in the bibliography
%    - Sort references by label (when citing)
%    - Set 2 names as max before invoking "et al." when citing
%    - Print all names in bibliography
\ExecuteBibliographyOptions{%
  firstinits=true,
  sortcites=true,
  maxcitenames=2,
  maxbibnames=100,
  hyperref=true,
  urldate=iso8601,
}

% Set author names (but only authors) in small caps (but only in the
% bibliography, not in the citations)
% http://tex.stackexchange.com/a/29862
% http://tex.stackexchange.com/a/30910
\AtBeginBibliography{%
  \renewcommand*{\mkbibnamelast}[1]{%
    \ifmknamesc{\textsc{#1}}{#1}%
  }%
}
\renewcommand*{\mkbibnameprefix}[1]{%
  \ifboolexpr{test {\ifmknamesc} and test {\ifuseprefix}}{%
    \textsc{#1}%
  }{%
    #1%
  }%
}
\def\ifmknamesc{%
  \ifboolexpr{%
    test {\ifcurrentname{labelname}}
    or test {\ifcurrentname{author}}
    or (test {\ifnameundef{author}} and test {\ifcurrentname{editor}})
  }%
}

% Customize appearance of chapter heading
\newcommand{\refname}{References}
\defbibheading{bibliography}[\refname]{%
  \chapter*{#1}%
  \addcontentsline{toc}{chapter}{\refname}%
  \markboth{#1}{#1}%
}

% Set URLs in smaller font in bibliography
\DeclareFieldFormat{url}{URL:~\small\url{#1}}

% Set ISBN, ISSN, and DOI fields in normal font
\DeclareFieldFormat{isbn}{ISBN:~#1}
\DeclareFieldFormat{issn}{ISSN:~#1}
\DeclareFieldFormat{doi}{DOI:~#1}

% Use "Doctoral thesis" instead of "PhD thesis"
\DefineBibliographyStrings{english}{%
  phdthesis={doctoral thesis},
}

% Prevent pagebreaks within entries
% http://tex.stackexchange.com/a/43275/2634
\patchcmd{\bibsetup}{\interlinepenalty=5000}{\interlinepenalty=10000}{}{}

% Remove terminating period from every bib entry
\renewcommand{\finentrypunct}{}
