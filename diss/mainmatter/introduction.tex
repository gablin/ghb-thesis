% Copyright (c) 2017, Gabriel Hjort Blindell <ghb@kth.se>
%
% This work is licensed under a Creative Commons 4.0 International License (see
% LICENSE file or visit <http://creativecommons.org/licenses/by/4.0/> for a copy
% of the license).

\chapter{Introduction}

\todo{write chapter outline}

\section{Background and Motivation}

\begin{figure}
  \centering%
  % Copyright (c) 2017, Gabriel Hjort Blindell <ghb@kth.se>
%
% This work is licensed under a Creative Commons 4.0 International License (see
% LICENSE file or visit <http://creativecommons.org/licenses/by/4.0/> for a copy
% of the license).
%
\begingroup%
\figureFont%
\newdimen\stageWidth%
\stageWidth=1.5cm%
\newdimen\stageHeight%
\stageHeight=8mm%
\newdimen\stageDist%
\stageDist=0.6\stageWidth%
\pgfdeclarelayer{background}%
\pgfsetlayers{background,main}%
\begin{tikzpicture}[%
    input/.style={%
      inner sep=0,
      outer sep=2pt,
      node distance=\stageDist,
      font=\strut,
    },
    output/.style={%
      input,
    },
    box/.style={%
      draw,
    },
    box/.append style={%
      thick,
    },
    compiler stage/.style={%
      box,
      inner sep=0,
      outer sep=0,
      minimum width=\stageWidth,
      minimum height=\stageHeight,
      node distance=\stageDist,
      fill=shade0,
    },
    compiler wrapper/.style={%
      box,
      inner xsep=0.5\stageDist,
      inner ysep=0.5\stageHeight,
      rounded corners=6pt,
      fill=shade2,
    },
    backend stage/.style={%
      compiler stage,
      minimum width=\stageWidth,
      minimum height=\stageHeight,
      node distance=0.5\stageDist,
    },
    backend wrapper/.style={%
      box,
      inner xsep=0.33\stageDist,
      inner ysep=0.33\stageHeight,
      rounded corners=3pt,
      fill=shade1,
    },
    flow/.style={%
      ->,
      thick,
    },
    label/.style={%
      inner sep=0,
      outer sep=3pt,
      node distance=0,
    },
    wrapper label/.style={%
      label,
      outer sep=1pt,
      font=\strut\bfseries,
    },
  ]

  \small

  % Compiler stages
  \node [input] (input) {program};
  \node [compiler stage, right=of input] (frontend) {frontend};
  \node [compiler stage, right=of frontend] (optimizer) {optimizer};
  \node [compiler stage, right=of optimizer] (backend) {backend};
  \node [output, right=of backend] (output) {assembly code};
  \begin{pgfonlayer}{background}
    \node [compiler wrapper, fit=(frontend) (optimizer) (backend)]
          (compiler-wrapper) {};
  \end{pgfonlayer}
  \node [wrapper label, above=of compiler-wrapper] {compiler};

  % Compiler execution flow
  \begin{scope}[flow]
    \draw (input) -- (frontend);
    \draw (frontend) -- node [label, above] {IR} (optimizer);
    \draw (optimizer) -- node [label, above] {IR} (backend);
    \draw (backend) -- (output);
  \end{scope}

  \footnotesize

  % Backend stages
  \node [backend stage, below=1.5\stageDist of backend] (regalloc) {%
    \begin{tabular}{c}
      register\\
      allocator
    \end{tabular}
  };
  \node [backend stage, left=of regalloc] (isel) {%
    \begin{tabular}{c}
      instruction\\
      selector
    \end{tabular}
  };
  \node [backend stage, right=of regalloc] (isched) {%
    \begin{tabular}{c}
      instruction\\
      scheduler
    \end{tabular}
  };
  \begin{pgfonlayer}{background}
    \node [backend wrapper, fit=(isel) (regalloc) (isched)]
          (backend-wrapper) {};
  \end{pgfonlayer}

  % Backstage execution flow
  \begin{scope}[flow]
    \draw ([xshift=-0.3\stageWidth] backend-wrapper.west) -- (isel);
    \draw (isel) -- (regalloc);
    \draw (regalloc) -- (isched);
    \draw (isched) -- ([xshift=0.3\stageWidth] backend-wrapper.east);
  \end{scope}
  \node [wrapper label, above=of backend-wrapper] {backend};
\end{tikzpicture}%
\endgroup%


  \caption{Overview of a typical compiler}
  \labelFigure{compiler-overview}
\end{figure}

A \gls!{compiler} is a tool that takes a \gls!{program}, written in some
programming language, as input and produces equivalent assembly code for a
specific \gls!{target machine} as output.
%
As shown in \refFigure{compiler-overview}, a compiler typically consists of
three parts:%
%
\begin{inlinelist}[itemjoin={; }, itemjoin*={; and}]
  \item a \gls!{frontend}, which performs syntactic and semantic analysis and
    transforms the program into an \gls!{IR}
  \item an \gls!{optimizer} (sometimes called \gls!{middle-end}), which performs
    target-independent optimizations
  \item a \gls!{backend}, which performs \gls{code generation}
\end{inlinelist}.
%
\Gls{code generation} in turn consists of several tasks, of which three
typically are most promiment:%
%
\begin{inlinelist}[itemjoin={; }, itemjoin*={; and}]
  \item \gls!{instruction selection}, where \glspl{instruction} implementing the
    given \gls{program} are selected
  \item \gls!{register allocation}, where \gls{virtual.temp} \glspl{temporary}
    are assigned to \glspl{register}
  \item \gls!{instruction scheduling}, where \glspl{instruction} are reordered
    to increase instruction-level parallelism
\end{inlinelist}.
%
This dissertation focuses primarily on the first problem by introducing a
combinatorial optimization model for \gls{instruction selection} which is
simpler, more flexible, and produces better code than traditional approaches and
captures more features than existing combinatorial optimization models.
%
The dissertation also touches upon the other two \gls{code generation} problems
by proposing ideas on how to extend the model to integrate \gls{instruction
  scheduling} and \gls{register allocation}.

\paragraph{Instruction Selection}






Selecting \glspl{instruction} for a given \gls{program} and \gls{target machine}
is an optimization problem, meaning there often exist more than one
\gls{solution} for a given \gls{program} which may result in code where quality
differs significantly.
%
For example, depending on the hardware the efficiency of two sets of selected
instructions may differ by as much as two orders of
magnitude~\cite{ZivojnovicEtAl:1994}.
%
Finding the optimal solution, however, is an NP-complete
problem~\mbox{\cite{BrunoSethi:1976,KoesGoldstein:2008}}, and the problems also
interact with one another.
%
Consequently, most compilers solve each problem separately using greedy
heuristics.

Several optimal approaches exist, 

\todo{optimal approach exist, but they have limitations}

\subsection{A Motivating Example}

\begin{filecontents*}{isel-gcmotion-example.c}
int i = 0;
while (i < N) {
  int c = A[i] + B[i];
  if (MAX < c) c = MAX;
  C[i] = c;
  i++;
}
\end{filecontents*}

\begin{figure}
  \centering%
  \subcaptionbox{C code\labelFigure{isel-gcmotion-example-c}}%
                {\lstinputlisting[language=c]{isel-gcmotion-example.c}}%
  \hspace{5mm}%
  \subcaptionbox{Corresponding \gls{IR} and control-flow graph%
                 \labelFigure{example-program-ir}}%
                [64mm]%
                {% Copyright (c) 2017, Gabriel Hjort Blindell <ghb@kth.se>
%
% This work is licensed under a Creative Commons 4.0 International License (see
% LICENSE file or visit <http://creativecommons.org/licenses/by/4.0/> for a copy
% of the license).
%
\begingroup%
\footnotesize%
\def\nodeDist{12pt}%
\pgfdeclarelayer{background}%
\pgfsetlayers{background,main}%
\begin{tikzpicture}[
    every node/.style={
      draw,
      fill=white,
      very thick,
      minimum size=0,
      inner xsep=4pt,
      inner ysep=2.5pt,
      node distance=\nodeDist,
      font=\ttfamily,
    },
    flow/.style={
      ->,
      very thick,
      shorten <=-2pt,
    },
    tf label/.style={
      draw=none,
      fill=none,
      minimum size=0,
      inner xsep=4pt,
      node distance=0,
      font=\sffamily,
      auto,
    },
    block label/.style={
      tf label,
      inner xsep=0,
      inner ysep=1pt,
      font=\sffamily\bfseries,
    },
  ]

  \node (b1) {i $=$ 0};
  \node [below=of b1, inner xsep=1em] (b2) {if i $<$ N};
  \node [below=of b2] (b3)
        {%
          \begin{tabular}{@{}l@{}}
            \irTemp{1} = i $*$ 4\\
            \irTemp{2} = A + \irTemp{1}; \irTemp{3} = B + \irTemp{1}\\
            a = load \irTemp{2}; b = load \irTemp{3}\\
            c = a $+$ b\\
            if MAX $<$ c
          \end{tabular}%
        };
  \node [below right=1.25*\nodeDist and 0 of b3.south west] (b4) {c $=$ MAX};
  \node [below left= 1.25*\nodeDist and 0 of b3.south east] (b5)
        {%
          \begin{tabular}{@{}l@{}}
            \irTemp{4} = C $+$ \irTemp{1}\\
            store \irTemp{4}, c\\
            i = i $+$ 1
          \end{tabular}%
        };

  \begin{pgfonlayer}{background}
    \begin{scope}[flow]
      \draw (b1) -- (b2);
      \draw (b2) -- node [tf label] {T} (b3);
      \draw (b3) -- node [tf label, inner xsep=2pt, pos=0.025] {T} (b4);
      \draw (b3) -- node [tf label, inner xsep=2pt, pos=0.025, swap] {F} (b5);
      \draw (b4) -- (b4 -| b5.west);
      \draw [rounded corners=10pt]
            ([yshift=10pt] b5.south west)
            -|
            ([xshift=-10pt] b3.west)
            |-
            (b2);
      \draw (b2) -- node [tf label] {F} ([xshift=2*\nodeDist] b2.east);
    \end{scope}
  \end{pgfonlayer}

  \node [block label, left=of b1, inner xsep=2pt] {b1\hspace*{-1pt}};
  \foreach \i in {2, 3, 4} {
    \node [block label, above right=0 and 0 of b\i.north west] {b\i};
  }
  \node [block label, above left=0 and 0 of b5.north east] {b5};
\end{tikzpicture}%
\endgroup%
}%

  \caption[A program computing saturated sums]%
          {%
            A program computing the saturated sums of two arrays, where
            \cVar{A}, \cVar{B}, and \cVar{C} are integer arrays of equal lengths
            stored in memory, and \cVar{N} and \cVar{MAX} are constants
            representing the array length and the upper limit, respectively. An
            integer is assumed to be 4~bytes%
          }
  \labelFigure{isel-gcmotion-example}%
\end{figure}

\RefFigure{isel-gcmotion-example} shows a program that computes the saturated
sums of two integer arrays. In saturation arithmetic, the result of an
arithmetic operation will always stay within a range fixed by a minimum and
maximum value. If the operation would produce a value outside of this range,
then the value is set (``clamped'') to the closest limit, thus becoming
``saturated''.

Let us now perform \gls{instruction selection} on this program for a machine
which has an instruction capable of performing multiple add operations (a
so-called \gls!{SIMD} \gls{instruction}).

\todo{Give example showing the interaction between instruction selection and
  global code motion}

\todo{Give another example showing the interaction between instruction selection
  and block ordering}

\todo{Traditional approaches (how)}
\todo{Limitations (why)}
\todo{Combinatorial optimization (what)}
\todo{Combinatorial approaches (how)}
\todo{Limitations (why)}
\todo{Research goal (what)}

\section{Thesis Statement}

\begin{statement}
  \Glstext{constraint programming} is a flexible, practical, and competitive
  approach to \glstext{instruction selection}.
\end{statement}

\todo{Define ``flexible''}
\todo{Define ``practical''}
\todo{Define ``competitive''}

\section{Approach}

\todo{Overview figure}

\section{Contributions}

This dissertation makes six contributions to the areas of code generation and
constraint programming:
%
\begin{contributions}
  \item \labelContribution{survey}
    a comprehensive and structured survey that covers over four decades of
    research in \gls{instruction selection};
  \item \labelContribution{representations}
    a program and instruction representation that enables
    \begin{contributions}
      \item \labelContribution{rep-uniformity}
        uniform treatment of data- and control-flow \glspl{operation}, and
      \item \labelContribution{rep-complex-instructions}
        modeling and pattern matching of complex \glspl{instruction} as
        \glspl{pattern}, and
      \item \labelContribution{rep-combining-problems}
        modeling of \gls{global.is} \gls{instruction selection} and \gls{global
          code motion} as a \gls{constraint model};
    \end{contributions}
  \item \labelContribution{constraint-model}
    a \gls{constraint model} and related transformations that, for the first
    time, integrates
    \begin{contributions}
      \item \labelContribution{cp-global-instruction-selection}
        \gls{global.is} \gls{instruction selection} with
      \item \labelContribution{cp-global-code-motion}
        \gls{global code motion},
    \end{contributions}
    and also integrates
    \begin{contributions}[resume]
      \item \labelContribution{cp-data-copying}
        \gls{data copying},
      \item \labelContribution{cp-block-ordering}
        \gls{block ordering}, and
      \item \labelContribution{cp-value-reuse}
        \gls{value reuse};
    \end{contributions}
  \item \labelContribution{solving-techniques}
    solving techniques that enable practical solving of the \gls{constraint
      model};
  \item \labelContribution{experiments}
    thorough experiments demonstrating that the approach scales to medium-sized
    programs and yields better code than traditional approaches; and
  \item \labelContribution{integration}
    an initial design showing how the \gls{constraint model} can be extended to
    integrate other code generation tasks, such as \gls{instruction scheduling}
    and \gls{register allocation}.
\end{contributions}
%
\refTable{contributions-per-chapter} shows in which chapters each contribution
is manifested and discussed further.

\begin{table}
  \centering%
  \begin{tabular}{c@{\qquad}*{12}{c}}
    \toprule
      \tabhead Chapter
    & \tabhead\refContribution{survey}
    & \multicolumn{3}{c}{\tabhead\refContribution{representations}}
    & \multicolumn{5}{c}{\tabhead\refContribution{constraint-model}}
    & \tabhead\refContribution{solving-techniques}
    & \tabhead\refContribution{experiments}
    & \tabhead\refContribution{integration} \\
    \cmidrule(lr){3-5}%
    \cmidrule(lr){6-10}%
    &
    & \tabhead\refContribution{rep-uniformity}
    & \tabhead\refContribution{rep-complex-instructions}
    & \tabhead\refContribution{rep-combining-problems}
    & \tabhead\refContribution{cp-global-instruction-selection}
    & \tabhead\refContribution{cp-global-code-motion}
    & \tabhead\refContribution{cp-data-copying}
    & \tabhead\refContribution{cp-block-ordering}
    & \tabhead\refContribution{cp-value-reuse}
    &
    &
    & \\
    \midrule
    \refChapter*{current-instruction-selection-techniques}
    & \supportYes
    & \supportNo
    & \supportNo
    & \supportNo
    & \supportNo
    & \supportNo
    & \supportNo
    & \supportNo
    & \supportNo
    & \supportNo
    & \supportNo
    & \supportNo \\
    \refChapter*{universal-representations}
    & \supportNo
    & \supportYes
    & \supportYes
    & \supportYes
    & \supportNo
    & \supportNo
    & \supportNo
    & \supportNo
    & \supportNo
    & \supportNo
    & \supportNo
    & \supportNo \\
    \refChapter*{pattern-matching}
    & \supportNo
    & \supportYes
    & \supportYes
    & \supportNo
    & \supportNo
    & \supportNo
    & \supportNo
    & \supportNo
    & \supportNo
    & \supportNo
    & \supportNo
    & \supportNo \\
    \refChapter*{modeling-global-instruction-selection}
    & \supportNo
    & \supportYes
    & \supportNo
    & \supportYes
    & \supportYes
    & \supportNo
    & \supportNo
    & \supportNo
    & \supportNo
    & \supportNo
    & \supportNo
    & \supportNo \\
    \refChapter*{modeling-global-code-motion}
    & \supportNo
    & \supportNo
    & \supportNo
    & \supportYes
    & \supportNo
    & \supportYes
    & \supportNo
    & \supportNo
    & \supportNo
    & \supportNo
    & \supportNo
    & \supportNo \\
    \refChapter*{modeling-data-copying}
    & \supportNo
    & \supportNo
    & \supportNo
    & \supportNo
    & \supportNo
    & \supportNo
    & \supportYes
    & \supportNo
    & \supportNo
    & \supportNo
    & \supportNo
    & \supportNo \\
    \refChapter*{modeling-block-ordering}
    & \supportNo
    & \supportNo
    & \supportNo
    & \supportNo
    & \supportNo
    & \supportNo
    & \supportNo
    & \supportYes
    & \supportNo
    & \supportNo
    & \supportNo
    & \supportNo \\
    \refChapter*{modeling-value-reuse}
    & \supportNo
    & \supportNo
    & \supportNo
    & \supportNo
    & \supportNo
    & \supportNo
    & \supportNo
    & \supportNo
    & \supportYes
    & \supportNo
    & \supportNo
    & \supportNo \\
    \refChapter*{solving-techniques}
    & \supportNo
    & \supportNo
    & \supportNo
    & \supportNo
    & \supportNo
    & \supportNo
    & \supportNo
    & \supportNo
    & \supportNo
    & \supportYes
    & \supportNo
    & \supportNo \\
    \refChapter*{comparison-against-the-state-of-the-art}
    & \supportNo
    & \supportNo
    & \supportNo
    & \supportNo
    & \supportNo
    & \supportNo
    & \supportNo
    & \supportNo
    & \supportNo
    & \supportNo
    & \supportYes
    & \supportNo \\
    \refChapter*{integrating-other-code-generation-tasks}
    & \supportNo
    & \supportNo
    & \supportNo
    & \supportNo
    & \supportNo
    & \supportNo
    & \supportNo
    & \supportNo
    & \supportNo
    & \supportNo
    & \supportNo
    & \supportYes \\
    \refChapter*{macro-expansion}
    & \supportYes
    & \supportNo
    & \supportNo
    & \supportNo
    & \supportNo
    & \supportNo
    & \supportNo
    & \supportNo
    & \supportNo
    & \supportNo
    & \supportNo
    & \supportNo \\
    \refChapter*{tree-covering}
    & \supportYes
    & \supportNo
    & \supportNo
    & \supportNo
    & \supportNo
    & \supportNo
    & \supportNo
    & \supportNo
    & \supportNo
    & \supportNo
    & \supportNo
    & \supportNo \\
    \refChapter*{dag-covering}
    & \supportYes
    & \supportNo
    & \supportNo
    & \supportNo
    & \supportNo
    & \supportNo
    & \supportNo
    & \supportNo
    & \supportNo
    & \supportNo
    & \supportNo
    & \supportNo \\
    \refChapter*{graph-covering}
    & \supportYes
    & \supportNo
    & \supportNo
    & \supportNo
    & \supportNo
    & \supportNo
    & \supportNo
    & \supportNo
    & \supportNo
    & \supportNo
    & \supportNo
    & \supportNo \\
    \bottomrule
  \end{tabular}

  \caption{Contributions per chapter}
  \labelTable{contributions-per-chapter}
\end{table}

\section{Publications}

This dissertation is based on material presented in the following publications:
%
\begin{publications}
  \item \labelPublication{survey-book}
    \fullcite{HjortBlindell:2016:Survey}.
  \item \labelPublication{cp-paper}
    \fullcite{HjortBlindellEtAl:2015:CP}.
  \item \labelPublication{cases-paper}
    \fullcite{HjortBlindellEtAl:2017:CASES}.
\end{publications}
%
\refTable{contributions-per-publication} shows the relation between the
contributions and the publications above.
%
\begin{table}
  \centering%
  \begin{tabular}{c@{\qquad}*{10}{c}}
    \toprule
      \tabhead Publication
    & \tabhead\refContribution{survey}
    & \tabhead\refContribution{representations}
    & \multicolumn{5}{c}{\tabhead\refContribution{constraint-model}}
    & \tabhead\refContribution{solving-techniques}
    & \tabhead\refContribution{experiments}
    & \tabhead\refContribution{integration} \\
    \cmidrule(lr){4-8}%
    &
    &
    & \tabhead\refContribution{cp-global-instruction-selection}
    & \tabhead\refContribution{cp-global-code-motion}
    & \tabhead\refContribution{cp-data-copying}
    & \tabhead\refContribution{cp-block-ordering}
    & \tabhead\refContribution{cp-value-reuse}
    &
    &
    & \\
    \midrule
    \refPublication{survey-book}
    & \supportYes
    & \supportNo
    & \supportNo
    & \supportNo
    & \supportNo
    & \supportNo
    & \supportNo
    & \supportNo
    & \supportNo
    & \supportNo \\
    \refPublication{cp-paper}
    & \supportNo
    & \supportYes
    & \supportYes
    & \supportYes
    & \supportYes
    & \supportYes
    & \supportNo
    & \supportNo
    & \supportYes
    & \supportNo \\
    \refPublication{cases-paper}
    & \supportNo
    & \supportNo
    & \supportNo
    & \supportNo
    & \supportNo
    & \supportNo
    & \supportYes
    & \supportYes
    & \supportYes
    & \supportNo \\
    \bottomrule
  \end{tabular}

  \caption{Contributions per publication}
  \labelTable{contributions-per-publication}
\end{table}

The author also participated in the following publications, which are out of
scope for the dissertation:

\begin{publications}[resume]
  \item \labelPublication{survey-report}
    \fullcite{HjortBlindell:2013:Survey}.
  \item \labelPublication{lctes}
    \fullcite{CastanedaLozanoEtAl:2014:LCTES}.
  \item \labelPublication{cc}
    \fullcite{CastanedaLozanoEtAl:2016:CC}.
  \item \labelPublication{fdl-2016}
    \fullcite{HjortBlindellEtAl:2016:FDL}.
\end{publications}
%
\refPublication{survey-report} is excluded as it is subsumed and extended by
\refPublication{survey-book}. \refPublication{lctes} and \refPublication{cc} are
excluded as they are only partially related to the dissertation (they apply
\gls{constraint programming} to solve \gls{register allocation} and
\gls{instruction scheduling}). \refPublication{fdl-2016} is excluded as it
belongs to a different topic entirely (high-level code generation for graphics
processors).

\section{Outline}

\todo{Describe chapters}
\todo{Add figure illustrating how to read the dissertation}
