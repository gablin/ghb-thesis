% Copyright (c) 2017, Gabriel Hjort Blindell <ghb@kth.se>
%
% This work is licensed under a Creative Commons 4.0 International License (see
% LICENSE file or visit <http://creativecommons.org/licenses/by/4.0/> for a copy
% of the license).

\chapter{Universal Representations}
\labelChapter{universal-representations}

\todo{write overview}



\section{Design Requirements}

To enable the \gls{constraint model} introduced later in this dissertation, we
require a \gls{graph}-based \gls{program} representation that fulfills the
following criteria:
%
\begin{enumerate}
  \item It must capture the data and control flow of an entire \gls{function}.
    %
    This is needed for modeling \gls{global.is} \gls{instruction selection}
    and for uniform selection of data and control \glspl{instruction}.
  \item \Glspl{operation}, both for data and control flow, must be explicitly
    represented as \glspl{node}.
    %
    This is needed for \gls{pattern matching}.
  \item Values produced and used by the \glspl{operation} must also be
    explicitly represented as \glspl{node}.
    %
    This is needed for modeling \gls{global code motion} and \gls{data copying}.
  \item The \gls{block} in which a particular \gls{operation} is to be performed
    must not be fixed.
    %
    This is needed for modeling \gls{global code motion}.
  \item The representation must be based on \gls{SSA}.
    %
    Not only is this practical as most \glspl{IR} used in modern
    \glspl{compiler} are already based on \gls{SSA}, it also helps in modeling
    \gls{global code motion}.
\end{enumerate}

\todo{discuss existing representation in relation to these requirements}



\section{Program Representation}

\todo{write}



\subsubsection{Proof of Semantic Preservation?}

\todo{write}



\section{Instruction Representation}

\todo{write}
