% Copyright (c) 2017, Gabriel Hjort Blindell <ghb@kth.se>
%
% This work is licensed under a Creative Commons 4.0 International License (see
% LICENSE file or visit <http://creativecommons.org/licenses/by/4.0/> for a copy
% of the license).

\chapter{Constraint Programming}
\labelChapter{constraint-programming}

This chapter describes \glsdesc{CP}, which is the method used in this
dissertation for modeling and solving the problems described in
\refChapter{introduction}.
%
\refSection{cp-overview} gives a brief overview (for a comprehensive overview of
\gls{CP}, see \cite{RossiEtAl:2006}).
%
\refSection{cp-modeling} describes how to describe problems as a \gls{constraint
  model}, and \refSection{cp-solving} describes the techniques applied in
solving these \glsplshort{constraint model}.



\section{Overview}
\labelSection{cp-overview}

As already mentioned, \glsdesc{CP} is a method for solving computationally hard
problems.
%
These problems are typically optimization problems, but the method can also be
applied to solve satisfaction problems.
%
In terms of modeling, \gls{CP} offers a higher level of abstraction than similar
methods such as \gls{IP} and \gls{SAT}~\cite{BiereEtAl:2009}.
%
For example, \gls{CP} provides dedicated constraints for capturing many
recurring problem structures that must be decomposed and reformulated in
\gls{IP} and \gls{SAT}~models.
%
This also makes \gls{CP} particularly suited for solving problems that appear in
\gls{instruction scheduling} and \gls{register allocation}, which is essential
for these tasks to be integrated with \gls{instruction selection}.



\section{Modeling}
\labelSection{cp-modeling}

\todo{write}



\section{Solving}
\labelSection{cp-solving}

\todo{write}



\paragraph{Propagation}

\todo{write}



\paragraph{Search}

\todo{write}



\paragraph{Implied Constraints and Dominance Breaking}

\todo{write}



\paragraph{Presolving}

\todo{write}
