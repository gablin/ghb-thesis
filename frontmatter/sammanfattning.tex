% Copyright (c) 2017, Gabriel Hjort Blindell <ghb@kth.se>
%
% This work is licensed under a Creative Commons 4.0 International License (see
% LICENSE file or visit <http://creativecommons.org/licenses/by/4.0/> for a copy
% of the license).

\begin{sammanfattning}
  \selectlanguage{swedish}
  %
  Inom kodgenerering v\"aljer instruktionsselektering (eng.\ \emph{instruction
    selection}) instruktioner f\"or att implementera ett givet program under
  kompilering, global kodf\"orflyttning (eng.\ \emph{global code motion})
  flyttar ber\"akningar fr\r{a}n en del av programmet till en annan, och
  blockl\"aggning (eng.\ \emph{block ordering}) placerar programblock i en
  sekventiell f\"oljd.
  %
  Denna avhandling introducerar en ny metod, kallad \emph{universell
    instruktionsselektering}, som integrerar global instruktionsselektering med
  global kodf\"orflyttning och blockl\"aggning.
  %
  Genom detta \r{a}tg\"ardar den begr\"ansningar hos befintliga
  instruktionsselekteringsmetoder som misslyckas med att utnyttja m\r{a}nga av
  instruktionerna som f\"orses av moderna processorer.

  F\"or att hantera den kombinatoriska naturen av dessa problem till\"ampas
  villkorsprogrammering, en teknik f\"or kombinatorisk optimering.
  %
  Metoden anv\"ander en innovativ model som \"ar enklare och mer flexibel
  j\"amf\"ort med metoderna som anv\"ands i moderna kompilatorer och som
  f\r{a}ngar viktiga s\"ardrag som ignoreras av andra kombinatoriska metoder.
  %
  Avhandlingen f\"oresl\r{a}r ocks\r{a} ut\"okningar av modellen f\"or att
  integrera instruktionsschemal\"aggning och registerallokering, tv\r{a} andra
  viktiga kodgenereringsproblem.

  Modellen m\"ojligg\"ors av en innovativ, grafbaserad representation som
  f\"orenar data- och kontrollfl\"ode f\"or hela funktioner.
  %
  Representationen \"ar avg\"orande f\"or att integrera instruktionsselektering
  med global kodf\"orflyttning och f\"or att modellera sofistikerade
  instruktioner, vars beteende omfattar b\r{a}de data- och kontrollfl\"ode, som
  grafer.

  Genom experimentell utv\"ardering visas att universell instruktionsselektering
  kan hantera arkitekturer med ett rikt instruktionsset och skalar upp till
  mediumstora funktioner.
  %
  F\"or dessa funktioner genererar den kod av motsvarande eller b\"attre
  kvalitet \"an den senaste tekniken.
  %
  Avhandlingen visar ocks\r{a} att det finns tillr\"ackligt med dataparallellism
  att utnyttja genom selektering av SIMD-instruktioner och att denna
  exploatering gynnas av global kodf\"orflyttning.
  %
  Med dessa resultat argumenteras f\"or att villkorsprogrammering \"ar en
  flexibel, praktisk, konkurrenskraftig, och ut\"okningsbar metod f\"or att
  kombinera global instruktionsselektering, global kodf\"orflyttning, och
  blockl\"aggning.
  %
  \selectlanguage{english}
\end{sammanfattning}
