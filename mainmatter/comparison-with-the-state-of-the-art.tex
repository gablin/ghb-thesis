% Copyright (c) 2017, Gabriel Hjort Blindell <ghb@kth.se>
%
% This work is licensed under a Creative Commons 4.0 International License (see
% LICENSE file or visit <http://creativecommons.org/licenses/by/4.0/> for a copy
% of the license).

\chapter{Comparison with the State of the Art}
\labelChapter{comparison-with-the-state-of-the-art}

\todo{write outline}


\section{Unison \versus LLVM}
\labelSection{comparison-unison-vs-llvm}

We evaluate the impact of our approach by comparing the cost (that is, the total
number of cycles, as described in \refSection{cm-objective-function} on
\refPageOfSection{cm-objective-function}) of \glspl{solution} produced by our
approach with the \glspl{solution} produced by \mbox{\gls{LLVM} 3.8}, an
existing state-of-the-art \gls{compiler}.

The experimental setup is the same as in
\refSection{cm-alt-values-experimental-evaluation} on
\refPageOfSection{cm-alt-values-experimental-evaluation} with three exceptions.
%
First, when filtering we remove all \glspl{function} that have less than less
than \num{50}~\gls{LLVM} \gls{IR} \glspl{instruction} -- anything smaller will
most likely not have enough data parallelism for selection of \gls{SIMD.instr}
\glspl{instruction} -- and greater than \num{150}~\glspl{instruction} --
anything larger will lead to unreasonably long experiment runtimes.
%
To increase the amount of data parallelism, we also remove all \glspl{function}
not containing at least two addition, subtraction, logical and, or logical or
\glspl{instruction}.
%
This leaves a pool of \num{221}~\glspl{function}, from which
\num{20}~\glspl{function} are sampled using the same method as before.
%
Second, when solving we apply time limit of \SI{600}{\s} to the \gls{constraint
  solver}.
%
For any given \gls{function}, the last \gls{solution} found is considered
optimal if and only if the \glsshort{constraint solver} has finished its
execution within the time limit.
%
Third, we apply the cost for the \gls{solution} computed by \gls{LLVM} for a
given \gls{function} as the upper bound for the \gls{cost variable} in the
corresponding instance of the \gls{constraint model}.

\newcommand{\UnisonVsLlvmHexagonFiveCyclesSpeedupSimpleNameAmean}{}
\newcommand{\UnisonVsLlvmHexagonFiveCyclesSpeedupSimpleNameGmean}{}
\newcommand{\UnisonVsLlvmHexagonFiveCyclesSpeedupSimpleNameMedian}{}
\newcommand{\UnisonVsLlvmHexagonFiveCyclesSpeedupSimpleNameMin}{}
\newcommand{\UnisonVsLlvmHexagonFiveCyclesSpeedupSimpleNameMax}{}
\newcommand{\UnisonVsLlvmHexagonFiveCyclesSpeedupSolutionFoundAvgAmean}{0.75}
\newcommand{\UnisonVsLlvmHexagonFiveCyclesSpeedupSolutionFoundAvgGmean}{}
\newcommand{\UnisonVsLlvmHexagonFiveCyclesSpeedupSolutionFoundAvgMedian}{1.0}
\newcommand{\UnisonVsLlvmHexagonFiveCyclesSpeedupSolutionFoundAvgMin}{0.0}
\newcommand{\UnisonVsLlvmHexagonFiveCyclesSpeedupSolutionFoundAvgMax}{1.0}
\newcommand{\UnisonVsLlvmHexagonFiveCyclesSpeedupCyclesAvgAmean}{9943.2666666666664}
\newcommand{\UnisonVsLlvmHexagonFiveCyclesSpeedupCyclesAvgGmean}{}
\newcommand{\UnisonVsLlvmHexagonFiveCyclesSpeedupCyclesAvgMedian}{8815.0}
\newcommand{\UnisonVsLlvmHexagonFiveCyclesSpeedupCyclesAvgMin}{1129.0}
\newcommand{\UnisonVsLlvmHexagonFiveCyclesSpeedupCyclesAvgMax}{26670.0}
\newcommand{\UnisonVsLlvmHexagonFiveCyclesSpeedupOptimalAvgAmean}{0.69999999999999996}
\newcommand{\UnisonVsLlvmHexagonFiveCyclesSpeedupOptimalAvgGmean}{}
\newcommand{\UnisonVsLlvmHexagonFiveCyclesSpeedupOptimalAvgMedian}{1.0}
\newcommand{\UnisonVsLlvmHexagonFiveCyclesSpeedupOptimalAvgMin}{0.0}
\newcommand{\UnisonVsLlvmHexagonFiveCyclesSpeedupOptimalAvgMax}{1.0}
\newcommand{\UnisonVsLlvmHexagonFiveCyclesSpeedupMatchingTimeAvgAmean}{3.7179785227874995}
\newcommand{\UnisonVsLlvmHexagonFiveCyclesSpeedupMatchingTimeAvgGmean}{}
\newcommand{\UnisonVsLlvmHexagonFiveCyclesSpeedupMatchingTimeAvgMedian}{1.9463023176250001}
\newcommand{\UnisonVsLlvmHexagonFiveCyclesSpeedupMatchingTimeAvgMin}{0.53676510200000005}
\newcommand{\UnisonVsLlvmHexagonFiveCyclesSpeedupMatchingTimeAvgMax}{27.999851821750003}
\newcommand{\UnisonVsLlvmHexagonFiveCyclesSpeedupLbCompTimeAvgAmean}{0.0}
\newcommand{\UnisonVsLlvmHexagonFiveCyclesSpeedupLbCompTimeAvgGmean}{}
\newcommand{\UnisonVsLlvmHexagonFiveCyclesSpeedupLbCompTimeAvgMedian}{0.0}
\newcommand{\UnisonVsLlvmHexagonFiveCyclesSpeedupLbCompTimeAvgMin}{0.0}
\newcommand{\UnisonVsLlvmHexagonFiveCyclesSpeedupLbCompTimeAvgMax}{0.0}
\newcommand{\UnisonVsLlvmHexagonFiveCyclesSpeedupDomProcTimeAvgAmean}{0.9032787621021271}
\newcommand{\UnisonVsLlvmHexagonFiveCyclesSpeedupDomProcTimeAvgGmean}{}
\newcommand{\UnisonVsLlvmHexagonFiveCyclesSpeedupDomProcTimeAvgMedian}{0.31207466125488281}
\newcommand{\UnisonVsLlvmHexagonFiveCyclesSpeedupDomProcTimeAvgMin}{0.054580271244049072}
\newcommand{\UnisonVsLlvmHexagonFiveCyclesSpeedupDomProcTimeAvgMax}{10.613963663578033}
\newcommand{\UnisonVsLlvmHexagonFiveCyclesSpeedupIllProcTimeAvgAmean}{1.1522458732128142}
\newcommand{\UnisonVsLlvmHexagonFiveCyclesSpeedupIllProcTimeAvgGmean}{}
\newcommand{\UnisonVsLlvmHexagonFiveCyclesSpeedupIllProcTimeAvgMedian}{0.45240148901939392}
\newcommand{\UnisonVsLlvmHexagonFiveCyclesSpeedupIllProcTimeAvgMin}{0.098498821258544922}
\newcommand{\UnisonVsLlvmHexagonFiveCyclesSpeedupIllProcTimeAvgMax}{11.328249752521515}
\newcommand{\UnisonVsLlvmHexagonFiveCyclesSpeedupRedunProcTimeAvgAmean}{0.45889250934123993}
\newcommand{\UnisonVsLlvmHexagonFiveCyclesSpeedupRedunProcTimeAvgGmean}{}
\newcommand{\UnisonVsLlvmHexagonFiveCyclesSpeedupRedunProcTimeAvgMedian}{0.14995300769805908}
\newcommand{\UnisonVsLlvmHexagonFiveCyclesSpeedupRedunProcTimeAvgMin}{0.039298951625823975}
\newcommand{\UnisonVsLlvmHexagonFiveCyclesSpeedupRedunProcTimeAvgMax}{5.2575463056564331}
\newcommand{\UnisonVsLlvmHexagonFiveCyclesSpeedupModelPrepTimeAvgAmean}{71.173000000000016}
\newcommand{\UnisonVsLlvmHexagonFiveCyclesSpeedupModelPrepTimeAvgGmean}{}
\newcommand{\UnisonVsLlvmHexagonFiveCyclesSpeedupModelPrepTimeAvgMedian}{28.10125}
\newcommand{\UnisonVsLlvmHexagonFiveCyclesSpeedupModelPrepTimeAvgMin}{7.0374999999999996}
\newcommand{\UnisonVsLlvmHexagonFiveCyclesSpeedupModelPrepTimeAvgMax}{648.78750000000014}
\newcommand{\UnisonVsLlvmHexagonFiveCyclesSpeedupSolvingTimeAvgAmean}{85.743000000000009}
\newcommand{\UnisonVsLlvmHexagonFiveCyclesSpeedupSolvingTimeAvgGmean}{}
\newcommand{\UnisonVsLlvmHexagonFiveCyclesSpeedupSolvingTimeAvgMedian}{3.5950000000000002}
\newcommand{\UnisonVsLlvmHexagonFiveCyclesSpeedupSolvingTimeAvgMin}{0.34999999999999998}
\newcommand{\UnisonVsLlvmHexagonFiveCyclesSpeedupSolvingTimeAvgMax}{608.4425}
\newcommand{\UnisonVsLlvmHexagonFiveCyclesSpeedupPrePlusSolvingTimeAvgAmean}{88.257417144656188}
\newcommand{\UnisonVsLlvmHexagonFiveCyclesSpeedupPrePlusSolvingTimeAvgGmean}{}
\newcommand{\UnisonVsLlvmHexagonFiveCyclesSpeedupPrePlusSolvingTimeAvgMedian}{4.5218974006175996}
\newcommand{\UnisonVsLlvmHexagonFiveCyclesSpeedupPrePlusSolvingTimeAvgMin}{0.58892790079116819}
\newcommand{\UnisonVsLlvmHexagonFiveCyclesSpeedupPrePlusSolvingTimeAvgMax}{635.64225972175598}
\newcommand{\UnisonVsLlvmHexagonFiveCyclesSpeedupTotalTimeAvgAmean}{91.975395667443692}
\newcommand{\UnisonVsLlvmHexagonFiveCyclesSpeedupTotalTimeAvgGmean}{}
\newcommand{\UnisonVsLlvmHexagonFiveCyclesSpeedupTotalTimeAvgMedian}{5.8637808412425994}
\newcommand{\UnisonVsLlvmHexagonFiveCyclesSpeedupTotalTimeAvgMin}{1.1256930027911682}
\newcommand{\UnisonVsLlvmHexagonFiveCyclesSpeedupTotalTimeAvgMax}{663.64211154350596}
\newcommand{\UnisonVsLlvmHexagonFiveCyclesSpeedupCyclesCvAmean}{0.0}
\newcommand{\UnisonVsLlvmHexagonFiveCyclesSpeedupCyclesCvGmean}{}
\newcommand{\UnisonVsLlvmHexagonFiveCyclesSpeedupCyclesCvMedian}{0.0}
\newcommand{\UnisonVsLlvmHexagonFiveCyclesSpeedupCyclesCvMin}{0.0}
\newcommand{\UnisonVsLlvmHexagonFiveCyclesSpeedupCyclesCvMax}{0.0}
\newcommand{\UnisonVsLlvmHexagonFiveCyclesSpeedupLbCompTimeCvAmean}{0.0}
\newcommand{\UnisonVsLlvmHexagonFiveCyclesSpeedupLbCompTimeCvGmean}{}
\newcommand{\UnisonVsLlvmHexagonFiveCyclesSpeedupLbCompTimeCvMedian}{0.0}
\newcommand{\UnisonVsLlvmHexagonFiveCyclesSpeedupLbCompTimeCvMin}{0.0}
\newcommand{\UnisonVsLlvmHexagonFiveCyclesSpeedupLbCompTimeCvMax}{0.0}
\newcommand{\UnisonVsLlvmHexagonFiveCyclesSpeedupDomProcTimeCvAmean}{0.0085684034418129548}
\newcommand{\UnisonVsLlvmHexagonFiveCyclesSpeedupDomProcTimeCvGmean}{}
\newcommand{\UnisonVsLlvmHexagonFiveCyclesSpeedupDomProcTimeCvMedian}{0.0057897518211587964}
\newcommand{\UnisonVsLlvmHexagonFiveCyclesSpeedupDomProcTimeCvMin}{0.0024640971455944954}
\newcommand{\UnisonVsLlvmHexagonFiveCyclesSpeedupDomProcTimeCvMax}{0.032228536959813321}
\newcommand{\UnisonVsLlvmHexagonFiveCyclesSpeedupIllProcTimeCvAmean}{0.0066656504854601673}
\newcommand{\UnisonVsLlvmHexagonFiveCyclesSpeedupIllProcTimeCvGmean}{}
\newcommand{\UnisonVsLlvmHexagonFiveCyclesSpeedupIllProcTimeCvMedian}{0.0061107179999732213}
\newcommand{\UnisonVsLlvmHexagonFiveCyclesSpeedupIllProcTimeCvMin}{0.0011263783148059748}
\newcommand{\UnisonVsLlvmHexagonFiveCyclesSpeedupIllProcTimeCvMax}{0.020950887628813971}
\newcommand{\UnisonVsLlvmHexagonFiveCyclesSpeedupRedunProcTimeCvAmean}{0.0074626067322486169}
\newcommand{\UnisonVsLlvmHexagonFiveCyclesSpeedupRedunProcTimeCvGmean}{}
\newcommand{\UnisonVsLlvmHexagonFiveCyclesSpeedupRedunProcTimeCvMedian}{0.0064914664813842593}
\newcommand{\UnisonVsLlvmHexagonFiveCyclesSpeedupRedunProcTimeCvMin}{0.0021052956721625206}
\newcommand{\UnisonVsLlvmHexagonFiveCyclesSpeedupRedunProcTimeCvMax}{0.029333112676738573}
\newcommand{\UnisonVsLlvmHexagonFiveCyclesSpeedupModelPrepTimeCvAmean}{0.0072537383106736925}
\newcommand{\UnisonVsLlvmHexagonFiveCyclesSpeedupModelPrepTimeCvGmean}{}
\newcommand{\UnisonVsLlvmHexagonFiveCyclesSpeedupModelPrepTimeCvMedian}{0.0076420315293359246}
\newcommand{\UnisonVsLlvmHexagonFiveCyclesSpeedupModelPrepTimeCvMin}{0.0019258808082683479}
\newcommand{\UnisonVsLlvmHexagonFiveCyclesSpeedupModelPrepTimeCvMax}{0.013511847677926439}
\newcommand{\UnisonVsLlvmHexagonFiveCyclesSpeedupSolvingTimeCvAmean}{0.0043932066411355494}
\newcommand{\UnisonVsLlvmHexagonFiveCyclesSpeedupSolvingTimeCvGmean}{}
\newcommand{\UnisonVsLlvmHexagonFiveCyclesSpeedupSolvingTimeCvMedian}{0.0037449756836741067}
\newcommand{\UnisonVsLlvmHexagonFiveCyclesSpeedupSolvingTimeCvMin}{0.0}
\newcommand{\UnisonVsLlvmHexagonFiveCyclesSpeedupSolvingTimeCvMax}{0.012191496227354279}
\newcommand{\UnisonVsLlvmHexagonFiveCyclesSpeedupPrePlusSolvingTimeCvAmean}{0.0035662398816440335}
\newcommand{\UnisonVsLlvmHexagonFiveCyclesSpeedupPrePlusSolvingTimeCvGmean}{}
\newcommand{\UnisonVsLlvmHexagonFiveCyclesSpeedupPrePlusSolvingTimeCvMedian}{0.0030719452472310151}
\newcommand{\UnisonVsLlvmHexagonFiveCyclesSpeedupPrePlusSolvingTimeCvMin}{0.00044192935097854975}
\newcommand{\UnisonVsLlvmHexagonFiveCyclesSpeedupPrePlusSolvingTimeCvMax}{0.0083459802243725494}
\newcommand{\UnisonVsLlvmHexagonFiveCyclesSpeedupTotalTimeCvAmean}{0.0027273228329425352}
\newcommand{\UnisonVsLlvmHexagonFiveCyclesSpeedupTotalTimeCvGmean}{}
\newcommand{\UnisonVsLlvmHexagonFiveCyclesSpeedupTotalTimeCvMedian}{0.0023105592026035574}
\newcommand{\UnisonVsLlvmHexagonFiveCyclesSpeedupTotalTimeCvMin}{0.00043944262607663255}
\newcommand{\UnisonVsLlvmHexagonFiveCyclesSpeedupTotalTimeCvMax}{0.00811316773429345}
\newcommand{\UnisonVsLlvmHexagonFiveCyclesSpeedupLlvmCyclesAmean}{8929.2999999999993}
\newcommand{\UnisonVsLlvmHexagonFiveCyclesSpeedupLlvmCyclesGmean}{}
\newcommand{\UnisonVsLlvmHexagonFiveCyclesSpeedupLlvmCyclesMedian}{8322.0}
\newcommand{\UnisonVsLlvmHexagonFiveCyclesSpeedupLlvmCyclesMin}{1088}
\newcommand{\UnisonVsLlvmHexagonFiveCyclesSpeedupLlvmCyclesMax}{26857}
\newcommand{\UnisonVsLlvmHexagonFiveCyclesSpeedupCyclesZeroCenteredSpeedupAmean}{n/a}
\newcommand{\UnisonVsLlvmHexagonFiveCyclesSpeedupCyclesZeroCenteredSpeedupGmean}{n/a}
\newcommand{\UnisonVsLlvmHexagonFiveCyclesSpeedupCyclesZeroCenteredSpeedupMedian}{0.0084238028740033343}
\newcommand{\UnisonVsLlvmHexagonFiveCyclesSpeedupCyclesZeroCenteredSpeedupMin}{0.0009075439591605219}
\newcommand{\UnisonVsLlvmHexagonFiveCyclesSpeedupCyclesZeroCenteredSpeedupMax}{0.18067026141560302}
\newcommand{\UnisonVsLlvmHexagonFiveCyclesSpeedupCyclesRegularSpeedupAmean}{n/a}
\newcommand{\UnisonVsLlvmHexagonFiveCyclesSpeedupCyclesRegularSpeedupGmean}{1.0342722554191712}
\newcommand{\UnisonVsLlvmHexagonFiveCyclesSpeedupCyclesRegularSpeedupMedian}{1.0084238028740033}
\newcommand{\UnisonVsLlvmHexagonFiveCyclesSpeedupCyclesRegularSpeedupMin}{1.0009075439591606}
\newcommand{\UnisonVsLlvmHexagonFiveCyclesSpeedupCyclesRegularSpeedupMax}{1.180670261415603}
\newcommand{\UnisonVsLlvmHexagonFiveCyclesSpeedupCyclesRegularSpeedupCiAmean}{n/a}
\newcommand{\UnisonVsLlvmHexagonFiveCyclesSpeedupCyclesRegularSpeedupCiGmean}{n/a}
\newcommand{\UnisonVsLlvmHexagonFiveCyclesSpeedupCyclesRegularSpeedupCiMedian}{n/a}
\newcommand{\UnisonVsLlvmHexagonFiveCyclesSpeedupCyclesRegularSpeedupCiMin}{1.012741267981361}
\newcommand{\UnisonVsLlvmHexagonFiveCyclesSpeedupCyclesRegularSpeedupCiMax}{1.0608074960839522}


\begin{figure}
  \centering%
  \maxsizebox{\textwidth}{!}{%
    \trimBarchartPlot{%
      \input{\expDir/unison-vs-llvm-hexagon5-cycles-speedup.plot}%
    }%
  }

  \caption[Plot for evaluating the impact of our approach on code quality]%
          {%
            Normalized solution costs for two pattern sets: one without SIMD
            instructions (baseline), and another with such instruction
            (subject).
            %
            GMI:~\printGMI{%
              \UnisonVsLlvmHexagonFiveCyclesSpeedupCyclesRegularSpeedupGmean%
            },
            CI:~\printGMICI{%
              \UnisonVsLlvmHexagonFiveCyclesSpeedupCyclesRegularSpeedupCiMin%
            }{%
              \UnisonVsLlvmHexagonFiveCyclesSpeedupCyclesRegularSpeedupCiMax%
            }.
            %
            The constraint model uses an upper bound computed by LLVM.
            %
            \Glspl{function} whose bars are marked with two dots are those
            for which the \gls{subject} fails to find the optimal solution%
          }
  \labelFigure{unison-vs-llvm-cycles-plot}
\end{figure}

\RefFigure{unison-vs-llvm-cycles-plot} shows the normalized \gls{solution} costs
for our approach, with \gls{LLVM} as \gls{baseline} and our approach as
\gls{subject}.
%
The costs range from
\printCycles{\UnisonVsLlvmHexagonFiveCyclesSpeedupCyclesAvgMin} to
\printCycles{\UnisonVsLlvmHexagonFiveCyclesSpeedupCyclesAvgMax}, with a maximum
coefficient of variation of
\num{\UnisonVsLlvmHexagonFiveCyclesSpeedupCyclesCvMax}.
%
The solving times range from
\printSolvingTime{\UnisonVsLlvmHexagonFiveCyclesSpeedupPrePlusSolvingTimeAvgMin}
to
\printSolvingTime{\UnisonVsLlvmHexagonFiveCyclesSpeedupPrePlusSolvingTimeAvgMax}
with a \gls{CV} of
\num{\UnisonVsLlvmHexagonFiveCyclesSpeedupPrePlusSolvingTimeCvMax}.
%
As the \gls{GMI} is \printGMI{%
  \UnisonVsLlvmHexagonFiveCyclesSpeedupCyclesRegularSpeedupGmean%
} with \gls{CI}~\printGMICI{%
  \UnisonVsLlvmHexagonFiveCyclesSpeedupCyclesRegularSpeedupCiMin%
}{%
  \UnisonVsLlvmHexagonFiveCyclesSpeedupCyclesRegularSpeedupCiMax%
}, we see that our approach produces \glspl{solution} with significantly lesser
cost than those produced by \gls{LLVM}.
%
\todo{case study}
%
Hence we conclude that our approach generates code of equal or better quality
compared to the state of the art.


\section{Selection of SIMD instructions}
\labelSection{comparison-with-or-without-simds}

\def\patternSetA{\textsc{i}}
\def\patternSetB{\textsc{ii}}

We evaluate the impact of being able to select \gls{SIMD.instr}
\glspl{instruction} by comparing the cost of \glspl{solution} produced from two
\glspl{pattern set}: one derived from \gls{Hexagon} without \gls{SIMD.instr}
\glspl{instruction}, and another with such \glspl{instruction}.
%
We refer to these as \glspl{pattern set}~\patternSetA{} and~\patternSetB,
respectively.

The experimental setup is the same as in \refSection{comparison-unison-vs-llvm}
with two exceptions.
%
First, when filtering we again remove all \glspl{function} that have less than
less than \num{50}~\gls{LLVM} \gls{IR} \glspl{instruction} -- anything smaller
will most likely not have enough data parallelism for selection of
\gls{SIMD.instr} \glspl{instruction} -- and greater than
\num{150}~\glspl{instruction} -- anything larger will lead to unreasonably long
experiment runtimes.
%
But to increase the chance of data parallelism, we also remove all
\glspl{function} not containing at least two addition, subtraction, logical and,
or logical or \glspl{instruction}.
%
This leaves a pool of \num{221}~\glspl{function}, from which
\num{20}~\glspl{function} are sampled using the same method as before.
%
Second, we do not apply an upper bound in this case as that may prevent
interesting \gls{solution} that make use of \gls{SIMD.instr} \gls{instruction}.

\newcommand{\SimdVsWithoutCyclesSpeedupSimpleNameAmean}{}
\newcommand{\SimdVsWithoutCyclesSpeedupSimpleNameGmean}{}
\newcommand{\SimdVsWithoutCyclesSpeedupSimpleNameMedian}{}
\newcommand{\SimdVsWithoutCyclesSpeedupSimpleNameMin}{}
\newcommand{\SimdVsWithoutCyclesSpeedupSimpleNameMax}{}
\newcommand{\SimdVsWithoutCyclesSpeedupSolutionFoundAvgAmean}{1.0}
\newcommand{\SimdVsWithoutCyclesSpeedupSolutionFoundAvgGmean}{}
\newcommand{\SimdVsWithoutCyclesSpeedupSolutionFoundAvgMedian}{1.0}
\newcommand{\SimdVsWithoutCyclesSpeedupSolutionFoundAvgMin}{1.0}
\newcommand{\SimdVsWithoutCyclesSpeedupSolutionFoundAvgMax}{1.0}
\newcommand{\SimdVsWithoutCyclesSpeedupCyclesAvgAmean}{6160.5500000000002}
\newcommand{\SimdVsWithoutCyclesSpeedupCyclesAvgGmean}{}
\newcommand{\SimdVsWithoutCyclesSpeedupCyclesAvgMedian}{5909.0}
\newcommand{\SimdVsWithoutCyclesSpeedupCyclesAvgMin}{390.0}
\newcommand{\SimdVsWithoutCyclesSpeedupCyclesAvgMax}{16963.0}
\newcommand{\SimdVsWithoutCyclesSpeedupOptimalAvgAmean}{0.94999999999999996}
\newcommand{\SimdVsWithoutCyclesSpeedupOptimalAvgGmean}{}
\newcommand{\SimdVsWithoutCyclesSpeedupOptimalAvgMedian}{1.0}
\newcommand{\SimdVsWithoutCyclesSpeedupOptimalAvgMin}{0.0}
\newcommand{\SimdVsWithoutCyclesSpeedupOptimalAvgMax}{1.0}
\newcommand{\SimdVsWithoutCyclesSpeedupMatchingTimeAvgAmean}{1.955276120325}
\newcommand{\SimdVsWithoutCyclesSpeedupMatchingTimeAvgGmean}{}
\newcommand{\SimdVsWithoutCyclesSpeedupMatchingTimeAvgMedian}{1.4190082572000002}
\newcommand{\SimdVsWithoutCyclesSpeedupMatchingTimeAvgMin}{0.67548939910000005}
\newcommand{\SimdVsWithoutCyclesSpeedupMatchingTimeAvgMax}{5.4709826762999993}
\newcommand{\SimdVsWithoutCyclesSpeedupLbCompTimeAvgAmean}{0.0}
\newcommand{\SimdVsWithoutCyclesSpeedupLbCompTimeAvgGmean}{}
\newcommand{\SimdVsWithoutCyclesSpeedupLbCompTimeAvgMedian}{0.0}
\newcommand{\SimdVsWithoutCyclesSpeedupLbCompTimeAvgMin}{0.0}
\newcommand{\SimdVsWithoutCyclesSpeedupLbCompTimeAvgMax}{0.0}
\newcommand{\SimdVsWithoutCyclesSpeedupDomProcTimeAvgAmean}{0.28117301464080813}
\newcommand{\SimdVsWithoutCyclesSpeedupDomProcTimeAvgGmean}{}
\newcommand{\SimdVsWithoutCyclesSpeedupDomProcTimeAvgMedian}{0.22162753343582153}
\newcommand{\SimdVsWithoutCyclesSpeedupDomProcTimeAvgMin}{0.075417351722717282}
\newcommand{\SimdVsWithoutCyclesSpeedupDomProcTimeAvgMax}{0.80465753078460689}
\newcommand{\SimdVsWithoutCyclesSpeedupIllProcTimeAvgAmean}{0.4775640988349914}
\newcommand{\SimdVsWithoutCyclesSpeedupIllProcTimeAvgGmean}{}
\newcommand{\SimdVsWithoutCyclesSpeedupIllProcTimeAvgMedian}{0.40084922313690186}
\newcommand{\SimdVsWithoutCyclesSpeedupIllProcTimeAvgMin}{0.13019657135009766}
\newcommand{\SimdVsWithoutCyclesSpeedupIllProcTimeAvgMax}{1.2684807538986207}
\newcommand{\SimdVsWithoutCyclesSpeedupRedunProcTimeAvgAmean}{0.16627510428428652}
\newcommand{\SimdVsWithoutCyclesSpeedupRedunProcTimeAvgGmean}{}
\newcommand{\SimdVsWithoutCyclesSpeedupRedunProcTimeAvgMedian}{0.13884462118148805}
\newcommand{\SimdVsWithoutCyclesSpeedupRedunProcTimeAvgMin}{0.052956461906433105}
\newcommand{\SimdVsWithoutCyclesSpeedupRedunProcTimeAvgMax}{0.41173338890075684}
\newcommand{\SimdVsWithoutCyclesSpeedupModelPrepTimeAvgAmean}{41.462350000000001}
\newcommand{\SimdVsWithoutCyclesSpeedupModelPrepTimeAvgGmean}{}
\newcommand{\SimdVsWithoutCyclesSpeedupModelPrepTimeAvgMedian}{25.892499999999998}
\newcommand{\SimdVsWithoutCyclesSpeedupModelPrepTimeAvgMin}{5.8010000000000002}
\newcommand{\SimdVsWithoutCyclesSpeedupModelPrepTimeAvgMax}{231.77600000000001}
\newcommand{\SimdVsWithoutCyclesSpeedupSolvingTimeAvgAmean}{76.068000000000012}
\newcommand{\SimdVsWithoutCyclesSpeedupSolvingTimeAvgGmean}{}
\newcommand{\SimdVsWithoutCyclesSpeedupSolvingTimeAvgMedian}{2.9139999999999997}
\newcommand{\SimdVsWithoutCyclesSpeedupSolvingTimeAvgMin}{0.30599999999999999}
\newcommand{\SimdVsWithoutCyclesSpeedupSolvingTimeAvgMax}{606.101}
\newcommand{\SimdVsWithoutCyclesSpeedupPrePlusSolvingTimeAvgAmean}{76.993012217760082}
\newcommand{\SimdVsWithoutCyclesSpeedupPrePlusSolvingTimeAvgGmean}{}
\newcommand{\SimdVsWithoutCyclesSpeedupPrePlusSolvingTimeAvgMedian}{3.7210775151252751}
\newcommand{\SimdVsWithoutCyclesSpeedupPrePlusSolvingTimeAvgMin}{0.64504172992706299}
\newcommand{\SimdVsWithoutCyclesSpeedupPrePlusSolvingTimeAvgMax}{608.58587167358405}
\newcommand{\SimdVsWithoutCyclesSpeedupTotalTimeAvgAmean}{78.948288338085092}
\newcommand{\SimdVsWithoutCyclesSpeedupTotalTimeAvgGmean}{}
\newcommand{\SimdVsWithoutCyclesSpeedupTotalTimeAvgMedian}{5.4664962529252747}
\newcommand{\SimdVsWithoutCyclesSpeedupTotalTimeAvgMin}{1.5170597081877319}
\newcommand{\SimdVsWithoutCyclesSpeedupTotalTimeAvgMax}{610.70832797298408}
\newcommand{\SimdVsWithoutCyclesSpeedupCyclesCvAmean}{0.0}
\newcommand{\SimdVsWithoutCyclesSpeedupCyclesCvGmean}{}
\newcommand{\SimdVsWithoutCyclesSpeedupCyclesCvMedian}{0.0}
\newcommand{\SimdVsWithoutCyclesSpeedupCyclesCvMin}{0.0}
\newcommand{\SimdVsWithoutCyclesSpeedupCyclesCvMax}{0.0}
\newcommand{\SimdVsWithoutCyclesSpeedupLbCompTimeCvAmean}{0.0}
\newcommand{\SimdVsWithoutCyclesSpeedupLbCompTimeCvGmean}{}
\newcommand{\SimdVsWithoutCyclesSpeedupLbCompTimeCvMedian}{0.0}
\newcommand{\SimdVsWithoutCyclesSpeedupLbCompTimeCvMin}{0.0}
\newcommand{\SimdVsWithoutCyclesSpeedupLbCompTimeCvMax}{0.0}
\newcommand{\SimdVsWithoutCyclesSpeedupDomProcTimeCvAmean}{0.012857329213125749}
\newcommand{\SimdVsWithoutCyclesSpeedupDomProcTimeCvGmean}{}
\newcommand{\SimdVsWithoutCyclesSpeedupDomProcTimeCvMedian}{0.010525659299158855}
\newcommand{\SimdVsWithoutCyclesSpeedupDomProcTimeCvMin}{0.0050058337990446242}
\newcommand{\SimdVsWithoutCyclesSpeedupDomProcTimeCvMax}{0.044813127695037355}
\newcommand{\SimdVsWithoutCyclesSpeedupIllProcTimeCvAmean}{0.0087105939941727363}
\newcommand{\SimdVsWithoutCyclesSpeedupIllProcTimeCvGmean}{}
\newcommand{\SimdVsWithoutCyclesSpeedupIllProcTimeCvMedian}{0.0063455948942023419}
\newcommand{\SimdVsWithoutCyclesSpeedupIllProcTimeCvMin}{0.0040309257956719991}
\newcommand{\SimdVsWithoutCyclesSpeedupIllProcTimeCvMax}{0.023963993549268585}
\newcommand{\SimdVsWithoutCyclesSpeedupRedunProcTimeCvAmean}{0.01408594463197381}
\newcommand{\SimdVsWithoutCyclesSpeedupRedunProcTimeCvGmean}{}
\newcommand{\SimdVsWithoutCyclesSpeedupRedunProcTimeCvMedian}{0.009950889592026637}
\newcommand{\SimdVsWithoutCyclesSpeedupRedunProcTimeCvMin}{0.004620537453705699}
\newcommand{\SimdVsWithoutCyclesSpeedupRedunProcTimeCvMax}{0.049724973485065722}
\newcommand{\SimdVsWithoutCyclesSpeedupModelPrepTimeCvAmean}{0.0082751982469220116}
\newcommand{\SimdVsWithoutCyclesSpeedupModelPrepTimeCvGmean}{}
\newcommand{\SimdVsWithoutCyclesSpeedupModelPrepTimeCvMedian}{0.0074264991328280476}
\newcommand{\SimdVsWithoutCyclesSpeedupModelPrepTimeCvMin}{0.004382858636059324}
\newcommand{\SimdVsWithoutCyclesSpeedupModelPrepTimeCvMax}{0.013408321484311392}
\newcommand{\SimdVsWithoutCyclesSpeedupSolvingTimeCvAmean}{0.011058272223471364}
\newcommand{\SimdVsWithoutCyclesSpeedupSolvingTimeCvGmean}{}
\newcommand{\SimdVsWithoutCyclesSpeedupSolvingTimeCvMedian}{0.0091441003978531936}
\newcommand{\SimdVsWithoutCyclesSpeedupSolvingTimeCvMin}{0.00031637044745634315}
\newcommand{\SimdVsWithoutCyclesSpeedupSolvingTimeCvMax}{0.039432543096335551}
\newcommand{\SimdVsWithoutCyclesSpeedupPrePlusSolvingTimeCvAmean}{0.008026940249999396}
\newcommand{\SimdVsWithoutCyclesSpeedupPrePlusSolvingTimeCvGmean}{}
\newcommand{\SimdVsWithoutCyclesSpeedupPrePlusSolvingTimeCvMedian}{0.0071226273917527937}
\newcommand{\SimdVsWithoutCyclesSpeedupPrePlusSolvingTimeCvMin}{0.00031469359036169145}
\newcommand{\SimdVsWithoutCyclesSpeedupPrePlusSolvingTimeCvMax}{0.022134646161960862}
\newcommand{\SimdVsWithoutCyclesSpeedupTotalTimeCvAmean}{0.005023657509666957}
\newcommand{\SimdVsWithoutCyclesSpeedupTotalTimeCvGmean}{}
\newcommand{\SimdVsWithoutCyclesSpeedupTotalTimeCvMedian}{0.004475246854342866}
\newcommand{\SimdVsWithoutCyclesSpeedupTotalTimeCvMin}{0.00031305275371001527}
\newcommand{\SimdVsWithoutCyclesSpeedupTotalTimeCvMax}{0.012054328528282381}
\newcommand{\SimdVsWithoutCyclesSpeedupBaselineSimpleNameAmean}{}
\newcommand{\SimdVsWithoutCyclesSpeedupBaselineSimpleNameGmean}{}
\newcommand{\SimdVsWithoutCyclesSpeedupBaselineSimpleNameMedian}{}
\newcommand{\SimdVsWithoutCyclesSpeedupBaselineSimpleNameMin}{}
\newcommand{\SimdVsWithoutCyclesSpeedupBaselineSimpleNameMax}{}
\newcommand{\SimdVsWithoutCyclesSpeedupBaselineSolutionFoundAvgAmean}{1.0}
\newcommand{\SimdVsWithoutCyclesSpeedupBaselineSolutionFoundAvgGmean}{}
\newcommand{\SimdVsWithoutCyclesSpeedupBaselineSolutionFoundAvgMedian}{1.0}
\newcommand{\SimdVsWithoutCyclesSpeedupBaselineSolutionFoundAvgMin}{1.0}
\newcommand{\SimdVsWithoutCyclesSpeedupBaselineSolutionFoundAvgMax}{1.0}
\newcommand{\SimdVsWithoutCyclesSpeedupBaselineCyclesAvgAmean}{6362.1000000000004}
\newcommand{\SimdVsWithoutCyclesSpeedupBaselineCyclesAvgGmean}{}
\newcommand{\SimdVsWithoutCyclesSpeedupBaselineCyclesAvgMedian}{5909.0}
\newcommand{\SimdVsWithoutCyclesSpeedupBaselineCyclesAvgMin}{390.0}
\newcommand{\SimdVsWithoutCyclesSpeedupBaselineCyclesAvgMax}{18963.0}
\newcommand{\SimdVsWithoutCyclesSpeedupBaselineOptimalAvgAmean}{0.94999999999999996}
\newcommand{\SimdVsWithoutCyclesSpeedupBaselineOptimalAvgGmean}{}
\newcommand{\SimdVsWithoutCyclesSpeedupBaselineOptimalAvgMedian}{1.0}
\newcommand{\SimdVsWithoutCyclesSpeedupBaselineOptimalAvgMin}{0.0}
\newcommand{\SimdVsWithoutCyclesSpeedupBaselineOptimalAvgMax}{1.0}
\newcommand{\SimdVsWithoutCyclesSpeedupBaselineMatchingTimeAvgAmean}{1.7509847168599997}
\newcommand{\SimdVsWithoutCyclesSpeedupBaselineMatchingTimeAvgGmean}{}
\newcommand{\SimdVsWithoutCyclesSpeedupBaselineMatchingTimeAvgMedian}{1.3751503890999999}
\newcommand{\SimdVsWithoutCyclesSpeedupBaselineMatchingTimeAvgMin}{0.65815778029999994}
\newcommand{\SimdVsWithoutCyclesSpeedupBaselineMatchingTimeAvgMax}{4.1755638625000007}
\newcommand{\SimdVsWithoutCyclesSpeedupBaselineLbCompTimeAvgAmean}{0.0}
\newcommand{\SimdVsWithoutCyclesSpeedupBaselineLbCompTimeAvgGmean}{}
\newcommand{\SimdVsWithoutCyclesSpeedupBaselineLbCompTimeAvgMedian}{0.0}
\newcommand{\SimdVsWithoutCyclesSpeedupBaselineLbCompTimeAvgMin}{0.0}
\newcommand{\SimdVsWithoutCyclesSpeedupBaselineLbCompTimeAvgMax}{0.0}
\newcommand{\SimdVsWithoutCyclesSpeedupBaselineDomProcTimeAvgAmean}{0.27489742517471316}
\newcommand{\SimdVsWithoutCyclesSpeedupBaselineDomProcTimeAvgGmean}{}
\newcommand{\SimdVsWithoutCyclesSpeedupBaselineDomProcTimeAvgMedian}{0.21316448450088502}
\newcommand{\SimdVsWithoutCyclesSpeedupBaselineDomProcTimeAvgMin}{0.075976967811584473}
\newcommand{\SimdVsWithoutCyclesSpeedupBaselineDomProcTimeAvgMax}{0.8065747737884521}
\newcommand{\SimdVsWithoutCyclesSpeedupBaselineIllProcTimeAvgAmean}{0.46612889409065239}
\newcommand{\SimdVsWithoutCyclesSpeedupBaselineIllProcTimeAvgGmean}{}
\newcommand{\SimdVsWithoutCyclesSpeedupBaselineIllProcTimeAvgMedian}{0.38459613323211672}
\newcommand{\SimdVsWithoutCyclesSpeedupBaselineIllProcTimeAvgMin}{0.13143212795257569}
\newcommand{\SimdVsWithoutCyclesSpeedupBaselineIllProcTimeAvgMax}{1.2814180135726929}
\newcommand{\SimdVsWithoutCyclesSpeedupBaselineRedunProcTimeAvgAmean}{0.16640672206878665}
\newcommand{\SimdVsWithoutCyclesSpeedupBaselineRedunProcTimeAvgGmean}{}
\newcommand{\SimdVsWithoutCyclesSpeedupBaselineRedunProcTimeAvgMedian}{0.13715589046478271}
\newcommand{\SimdVsWithoutCyclesSpeedupBaselineRedunProcTimeAvgMin}{0.052832317352294919}
\newcommand{\SimdVsWithoutCyclesSpeedupBaselineRedunProcTimeAvgMax}{0.40217461585998537}
\newcommand{\SimdVsWithoutCyclesSpeedupBaselineModelPrepTimeAvgAmean}{40.556549999999994}
\newcommand{\SimdVsWithoutCyclesSpeedupBaselineModelPrepTimeAvgGmean}{}
\newcommand{\SimdVsWithoutCyclesSpeedupBaselineModelPrepTimeAvgMedian}{25.824000000000005}
\newcommand{\SimdVsWithoutCyclesSpeedupBaselineModelPrepTimeAvgMin}{5.8100000000000005}
\newcommand{\SimdVsWithoutCyclesSpeedupBaselineModelPrepTimeAvgMax}{229.62299999999996}
\newcommand{\SimdVsWithoutCyclesSpeedupBaselineSolvingTimeAvgAmean}{62.710399999999993}
\newcommand{\SimdVsWithoutCyclesSpeedupBaselineSolvingTimeAvgGmean}{}
\newcommand{\SimdVsWithoutCyclesSpeedupBaselineSolvingTimeAvgMedian}{2.8570000000000002}
\newcommand{\SimdVsWithoutCyclesSpeedupBaselineSolvingTimeAvgMin}{0.307}
\newcommand{\SimdVsWithoutCyclesSpeedupBaselineSolvingTimeAvgMax}{606.08799999999997}
\newcommand{\SimdVsWithoutCyclesSpeedupBaselinePrePlusSolvingTimeAvgAmean}{63.617833041334151}
\newcommand{\SimdVsWithoutCyclesSpeedupBaselinePrePlusSolvingTimeAvgGmean}{}
\newcommand{\SimdVsWithoutCyclesSpeedupBaselinePrePlusSolvingTimeAvgMedian}{3.6290404148101802}
\newcommand{\SimdVsWithoutCyclesSpeedupBaselinePrePlusSolvingTimeAvgMin}{0.6421075563430787}
\newcommand{\SimdVsWithoutCyclesSpeedupBaselinePrePlusSolvingTimeAvgMax}{608.57816740322119}
\newcommand{\SimdVsWithoutCyclesSpeedupBaselineTotalTimeAvgAmean}{65.368817758194169}
\newcommand{\SimdVsWithoutCyclesSpeedupBaselineTotalTimeAvgGmean}{}
\newcommand{\SimdVsWithoutCyclesSpeedupBaselineTotalTimeAvgMedian}{5.2188554371101805}
\newcommand{\SimdVsWithoutCyclesSpeedupBaselineTotalTimeAvgMin}{1.4876764585645141}
\newcommand{\SimdVsWithoutCyclesSpeedupBaselineTotalTimeAvgMax}{610.67185656142124}
\newcommand{\SimdVsWithoutCyclesSpeedupBaselineCyclesCvAmean}{0.0}
\newcommand{\SimdVsWithoutCyclesSpeedupBaselineCyclesCvGmean}{}
\newcommand{\SimdVsWithoutCyclesSpeedupBaselineCyclesCvMedian}{0.0}
\newcommand{\SimdVsWithoutCyclesSpeedupBaselineCyclesCvMin}{0.0}
\newcommand{\SimdVsWithoutCyclesSpeedupBaselineCyclesCvMax}{0.0}
\newcommand{\SimdVsWithoutCyclesSpeedupBaselineLbCompTimeCvAmean}{0.0}
\newcommand{\SimdVsWithoutCyclesSpeedupBaselineLbCompTimeCvGmean}{}
\newcommand{\SimdVsWithoutCyclesSpeedupBaselineLbCompTimeCvMedian}{0.0}
\newcommand{\SimdVsWithoutCyclesSpeedupBaselineLbCompTimeCvMin}{0.0}
\newcommand{\SimdVsWithoutCyclesSpeedupBaselineLbCompTimeCvMax}{0.0}
\newcommand{\SimdVsWithoutCyclesSpeedupBaselineDomProcTimeCvAmean}{0.011429107079059784}
\newcommand{\SimdVsWithoutCyclesSpeedupBaselineDomProcTimeCvGmean}{}
\newcommand{\SimdVsWithoutCyclesSpeedupBaselineDomProcTimeCvMedian}{0.011360088447338743}
\newcommand{\SimdVsWithoutCyclesSpeedupBaselineDomProcTimeCvMin}{0.0056969362582002117}
\newcommand{\SimdVsWithoutCyclesSpeedupBaselineDomProcTimeCvMax}{0.023417863035505457}
\newcommand{\SimdVsWithoutCyclesSpeedupBaselineIllProcTimeCvAmean}{0.0087970839423916272}
\newcommand{\SimdVsWithoutCyclesSpeedupBaselineIllProcTimeCvGmean}{}
\newcommand{\SimdVsWithoutCyclesSpeedupBaselineIllProcTimeCvMedian}{0.008695892931095052}
\newcommand{\SimdVsWithoutCyclesSpeedupBaselineIllProcTimeCvMin}{0.0038229209741867036}
\newcommand{\SimdVsWithoutCyclesSpeedupBaselineIllProcTimeCvMax}{0.020746719577098699}
\newcommand{\SimdVsWithoutCyclesSpeedupBaselineRedunProcTimeCvAmean}{0.017284505009488547}
\newcommand{\SimdVsWithoutCyclesSpeedupBaselineRedunProcTimeCvGmean}{}
\newcommand{\SimdVsWithoutCyclesSpeedupBaselineRedunProcTimeCvMedian}{0.013001287425830253}
\newcommand{\SimdVsWithoutCyclesSpeedupBaselineRedunProcTimeCvMin}{0.0057474060957849087}
\newcommand{\SimdVsWithoutCyclesSpeedupBaselineRedunProcTimeCvMax}{0.031976416498450469}
\newcommand{\SimdVsWithoutCyclesSpeedupBaselineModelPrepTimeCvAmean}{0.0081678431336698783}
\newcommand{\SimdVsWithoutCyclesSpeedupBaselineModelPrepTimeCvGmean}{}
\newcommand{\SimdVsWithoutCyclesSpeedupBaselineModelPrepTimeCvMedian}{0.006958461195703698}
\newcommand{\SimdVsWithoutCyclesSpeedupBaselineModelPrepTimeCvMin}{0.0039248723755563892}
\newcommand{\SimdVsWithoutCyclesSpeedupBaselineModelPrepTimeCvMax}{0.013655722200225873}
\newcommand{\SimdVsWithoutCyclesSpeedupBaselineSolvingTimeCvAmean}{0.010315529177105782}
\newcommand{\SimdVsWithoutCyclesSpeedupBaselineSolvingTimeCvGmean}{}
\newcommand{\SimdVsWithoutCyclesSpeedupBaselineSolvingTimeCvMedian}{0.0086723312529763744}
\newcommand{\SimdVsWithoutCyclesSpeedupBaselineSolvingTimeCvMin}{0.00037932499559131967}
\newcommand{\SimdVsWithoutCyclesSpeedupBaselineSolvingTimeCvMax}{0.028165053291460372}
\newcommand{\SimdVsWithoutCyclesSpeedupBaselinePrePlusSolvingTimeCvAmean}{0.0078031339670925747}
\newcommand{\SimdVsWithoutCyclesSpeedupBaselinePrePlusSolvingTimeCvGmean}{}
\newcommand{\SimdVsWithoutCyclesSpeedupBaselinePrePlusSolvingTimeCvMedian}{0.007383235621767275}
\newcommand{\SimdVsWithoutCyclesSpeedupBaselinePrePlusSolvingTimeCvMin}{0.00037396782837074527}
\newcommand{\SimdVsWithoutCyclesSpeedupBaselinePrePlusSolvingTimeCvMax}{0.019558012472512477}
\newcommand{\SimdVsWithoutCyclesSpeedupBaselineTotalTimeCvAmean}{0.0052757321002978742}
\newcommand{\SimdVsWithoutCyclesSpeedupBaselineTotalTimeCvGmean}{}
\newcommand{\SimdVsWithoutCyclesSpeedupBaselineTotalTimeCvMedian}{0.0044072273323732432}
\newcommand{\SimdVsWithoutCyclesSpeedupBaselineTotalTimeCvMin}{0.00037750612907299177}
\newcommand{\SimdVsWithoutCyclesSpeedupBaselineTotalTimeCvMax}{0.016028415285590958}
\newcommand{\SimdVsWithoutCyclesSpeedupCyclesZeroCenteredSpeedupAmean}{n/a}
\newcommand{\SimdVsWithoutCyclesSpeedupCyclesZeroCenteredSpeedupGmean}{n/a}
\newcommand{\SimdVsWithoutCyclesSpeedupCyclesZeroCenteredSpeedupMedian}{0.0}
\newcommand{\SimdVsWithoutCyclesSpeedupCyclesZeroCenteredSpeedupMin}{-0.0}
\newcommand{\SimdVsWithoutCyclesSpeedupCyclesZeroCenteredSpeedupMax}{0.11790367269940459}
\newcommand{\SimdVsWithoutCyclesSpeedupCyclesRegularSpeedupAmean}{n/a}
\newcommand{\SimdVsWithoutCyclesSpeedupCyclesRegularSpeedupGmean}{1.0161049791727153}
\newcommand{\SimdVsWithoutCyclesSpeedupCyclesRegularSpeedupMedian}{1.0}
\newcommand{\SimdVsWithoutCyclesSpeedupCyclesRegularSpeedupMin}{1.0}
\newcommand{\SimdVsWithoutCyclesSpeedupCyclesRegularSpeedupMax}{1.1179036726994045}
\newcommand{\SimdVsWithoutCyclesSpeedupCyclesRegularSpeedupCiAmean}{n/a}
\newcommand{\SimdVsWithoutCyclesSpeedupCyclesRegularSpeedupCiGmean}{n/a}
\newcommand{\SimdVsWithoutCyclesSpeedupCyclesRegularSpeedupCiMedian}{n/a}
\newcommand{\SimdVsWithoutCyclesSpeedupCyclesRegularSpeedupCiMin}{1.0034477975382945}
\newcommand{\SimdVsWithoutCyclesSpeedupCyclesRegularSpeedupCiMax}{1.0318008464308406}


\begin{figure}
  \centering%
  \maxsizebox{\textwidth}{!}{%
    \trimBarchartPlot{%
      \input{\expDir/simd-vs-without-cycles-speedup.plot}%
    }%
  }

  \caption[Plot for evaluating the impact of SIMD instructions on code quality]%
          {%
            Normalized solution costs for two pattern sets: one without SIMD
            instructions (baseline), and another with such instruction
            (subject).
            %
            GMI:~\printGMI{%
              \SimdVsWithoutCyclesSpeedupCyclesRegularSpeedupGmean%
            },
            CI:~\printGMICI{%
              \SimdVsWithoutCyclesSpeedupCyclesRegularSpeedupCiMin%
            }{%
              \SimdVsWithoutCyclesSpeedupCyclesRegularSpeedupCiMax%
            }.
            %
            \Glspl{function} whose bars are marked with two dots are those
            for which the \gls{subject} fails to find the optimal solution%
          }
  \labelFigure{simd-vs-without-cycles-plot}
\end{figure}

\RefFigure{simd-vs-without-cycles-plot} shows the normalized \gls{solution}
costs for the two \glspl{pattern set} describe above, with \gls{pattern
  set}~\patternSetA{} as \gls{baseline} and \gls{pattern set}~\patternSetB{} as
\gls{subject}.
%
The costs range from
\printMinCycles{%
  \SimdVsWithoutCyclesSpeedupCyclesAvgMin,
  \SimdVsWithoutCyclesSpeedupBaselineCyclesAvgMin
} to
\printMaxCycles{%
  \SimdVsWithoutCyclesSpeedupCyclesAvgMax,
  \SimdVsWithoutCyclesSpeedupBaselineCyclesAvgMax
}, with a \gls{CV} of
\numMaxOf{
  \SimdVsWithoutCyclesSpeedupCyclesCvMax,
  \SimdVsWithoutCyclesSpeedupBaselineCyclesCvMax
}.
%
The solving times from
\printMinSolvingTime{%
  \SimdVsWithoutCyclesSpeedupPrePlusSolvingTimeAvgMin,
  \SimdVsWithoutCyclesSpeedupBaselinePrePlusSolvingTimeAvgMin
} to
\printMaxSolvingTime{%
  \SimdVsWithoutCyclesSpeedupPrePlusSolvingTimeAvgMax,
  \SimdVsWithoutCyclesSpeedupBaselinePrePlusSolvingTimeAvgMax
}, with a \gls{CV} of
\numMaxOf{
  \SimdVsWithoutCyclesSpeedupPrePlusSolvingTimeCvMax,
  \SimdVsWithoutCyclesSpeedupBaselinePrePlusSolvingTimeCvMax
}.
%
As the \gls{GMI} is \printGMI{%
  \SimdVsWithoutCyclesSpeedupCyclesRegularSpeedupGmean%
} with \gls{CI}~\printGMICI{%
  \SimdVsWithoutCyclesSpeedupCyclesRegularSpeedupCiMin%
}{%
  \SimdVsWithoutCyclesSpeedupCyclesRegularSpeedupCiMax%
}, we see that the \gls{pattern set}~\patternSetB{} yields \glspl{solution} with
significantly lesser cost than those yielded by \glsshort{pattern
  set}~\patternSetA.
%
The five cases with lesser cost ({\codeFont debug\_print\_str}, {\codeFont
  gl\_read\_alpha}, {\codeFont gx\_curve\_cursor}, {\codeFont mp\_dmul}, and
{\codeFont zero}), our approach is able to combine pairs of additions or
subtractions into \num{2}-way \gls{SIMD.instr} \glspl{instruction}.
%
In addition, in one of these cases ({\codeFont gl\_read\_alpha}) the additions
originally reside in different \glspl{block}, but due to \gls{global code
  motion} our approach is able to move the operations to the same block in order
to implement these using a single \gls{instruction}.
%
\todo{make figure to illustrate the effect}
%
Hence we conclude that there is sufficient data parallelism to be exploited
through selection of \gls{SIMD.instr} \glspl{instruction}, resulting in
significantly better code quality, and that this exploitation is benefitted from
\gls{global code motion}.


\section{Summary}
\labelSection{comparison-summary}

\todo{write}

